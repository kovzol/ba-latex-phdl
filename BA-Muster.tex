\documentclass[12pt,a4paper,oneside,naustrian]{amsbook}
\usepackage[T1]{fontenc}
\usepackage[utf8]{inputenc}
\setcounter{secnumdepth}{2}
\setcounter{tocdepth}{1}
\usepackage{parskip}
\usepackage{amsthm}
\usepackage{amssymb}
\usepackage{graphicx}
\usepackage{setspace}
\usepackage{natbib}

\onehalfspacing

\makeatletter

\pdfpageheight\paperheight
\pdfpagewidth\paperwidth


%%%%%%%%%%%%%%%%%%%%%%%%%%%%%% Textclass specific LaTeX commands.
\numberwithin{section}{chapter}
\numberwithin{equation}{section}
\numberwithin{figure}{section}
\theoremstyle{plain}
\newtheorem{thm}{\protect\theoremname}
\theoremstyle{plain}
\newtheorem{prop}[thm]{\protect\propositionname}

%%%%%%%%%%%%%%%%%%%%%%%%%%%%%% User specified LaTeX commands.
\usepackage{graphicx,lipsum}
\usepackage[total={20.9cm,29.7cm},
margin=2.5cm,top=1.5cm,includeheadfoot]{geometry}
\setlength{\footskip}{2\baselineskip}

\makeatother

\usepackage{babel}
\providecommand{\propositionname}{Satz}
\providecommand{\theoremname}{Theorem}

\begin{document}

\savegeometry{saved}
\leftmargin 0cm
\textwidth 16.7cm
\textheight 23cm
\topmargin -2.7cm
\oddsidemargin 0.1cm
\parindent 0pt

\def\title{4D Druck im Unterricht:\\ Fraktale}
\def\name{A.~B.~Conrad Dachsteiner}
\def\institute{Institut für \\Didaktik der Mathematik}
\def\supervisor{Dr.~Zoltán Kovács}
\def\assist{Dr.~Zsolt Lavicza}
\def\date{Juni 2025}

\thispagestyle{empty}
\def\ifundefined#1{\expandafter\ifx\csname#1\endcsname\relax}
\DeclareFontShape{OT1}{cmss}{m}{n}
  {<5><6><7><8><9><10><10.95><12><14.4><17.28><20.74><24.88><29.86><35.83><42.99><51.59><67><77.38>cmss10}{}
\DeclareFontShape{OT1}{cmss}{bx}{n}
  {<5><6><7><8><9><10><10.95><12><14.4><17.28><20.74><24.88><29.86><35.83><42.99><51.59><67><77.38>cmssbx10}{}
\makeatletter
\def\Huge{\@setfontsize\Huge{29.86pt}{36}}
\makeatother
\unitlength 1cm
\sffamily
\begin{picture}(16.7,0)\put(-1.0,-2.5){

\includegraphics[width=7.22cm]{PHDL}\hspace{6mm}\includegraphics[width=3.41cm]{PH-OOe}\hspace{13mm}\includegraphics[width=4.94cm]{JKU}

}
\put(12.5,-4.2){\begin{minipage}[t]{5cm}\footnotesize%
 Eingereicht von\\ {\bfseries\name}%
\vskip 4mm%
 Angefertigt am\\ {\bfseries\institute}%
\vskip 4mm%
 Beurteiler / Beurteilerin\\ {\bfseries\supervisor}%
\vskip 4mm%
 Mitbetreuung\\ {\bfseries\assist}%
\vskip 4mm%
\date
\end{minipage}}

\put(0,-25){\begin{minipage}[t]{3.9cm}\footnotesize%
{\bfseries Private P\"adagogische \\ Hochschule der \\ Di\"ozese Linz}\\
Salesianumweg 3\\ 4020 Linz, Österreich\\ phdl.at
\end{minipage}}

\put(6.5,-25){\begin{minipage}[t]{5cm}\footnotesize%
{\bfseries Pädagogische Hochschule\\ Oberösterreich}\\
Kaplanhofstraße 40 /\\ Huemerstrasse 3 - 5\\
4020 Linz, Österreich\\ ph-ooe.at
\end{minipage}}

\put(12.9,-25){\begin{minipage}[t]{3.9cm}\footnotesize%
{\bfseries JOHANNES KEPLER\\ UNIVERSITÄT LINZ}\\
Altenbergerstra{\ss}e 69\\ 4040 Linz, Österreich\\
www.jku.at\\ DVR 0093696
\end{minipage}}

\put(0,-15.2){\begin{minipage}[b]{15cm}\Huge\bfseries\title\end{minipage}}

\put(0,-18.3){\begin{minipage}[t]{12cm}%
 {\large Bachelorarbeit}%
 \vskip 2mm%
 zur Erlangung des akademischen Grades%
 \vskip 3mm%
 {\large Bachelor of Education}
 \vskip 3mm%
 im Bachelorstudium
\vskip 3mm%
{\large Lehramt Sekundarstufe (Allgemeinbildung)}
\end{minipage}}
\end{picture}
\eject
\rmfamily

\loadgeometry{saved}
\pagestyle{plain}
\newpage{}
\begin{center}
\vspace*{0.3\paperheight}
Für meine geliebte Mama
\par\end{center}

\begin{center}
\thispagestyle{empty}\newpage\tableofcontents{}
\par\end{center}

\chapter{Einführung}

Bla\ldots{} bla\ldots{} bla\ldots{} \lipsum[1-10]

\section{Wichtige Bemerkungen}

\lipsum[11-20]

\section{Weitere wichtige Bemerkungen}

\lipsum[21-30]

\chapter{Tiefes Wasser}

In diesem Kapitel\ldots{} blabla\ldots{} bla-bla\ldots{} \lipsum[31]

\section{Sehr wichtige Bemerkungen}

\lipsum[33-35]

Nun schließen wir darauf, dass
\[
\sum_{n=0}^{\infty}\frac{1}{n^{2}}=\frac{\pi^{2}}{6}.
\]
Unsere Bemerkungen können in dieser Form zusammengefasst:
\begin{prop}
Für jedes $k\in\mathbb{N}$, $k>1$ gilt

\[
\sum_{n=0}^{\infty}\frac{1}{n^{k}}<\infty.
\]

\begin{proof}
Trivial.
\end{proof}
\end{prop}

Ein weiterer Beweis befindet sich im \citet{Adams1980}.
\begin{thebibliography}{}
\bibitem[Adams(1980)]{Adams1980}Douglas Adams: \emph{Das Restaurant am Ende
des Universums} \emph{(The Restaurant at the End of the Universe)},
ISBN 3-453-14698-0. Originalveröffentlichung 1980. Veröffentlichung
der deutschen Übersetzung 1983.

\end{thebibliography}

\end{document}
